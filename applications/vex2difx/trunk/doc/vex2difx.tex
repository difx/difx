\documentclass[12pt]{article}

% Notes to potential editors:
% 1. Please don't change the line wrapping.  Exactly one sentence per line!
% 2. Update "date" and "version" below with each update
% 3. Notation : 
%       Program names and other text to be typed by user or returned by the computer in {\tt }
%       Variables or arguments in {\em } or in $<$ $>$


\usepackage[margin=0.75in,twoside]{geometry}
\usepackage{graphics}
\usepackage{float}
\usepackage{color}
\definecolor{darkblue}{rgb}{0,0.2,0.4}
\usepackage[colorlinks,linkcolor=darkblue,citecolor=blue,urlcolor=blue,pdftitle={How vex2difx works},pdfauthor={Walter Brisken}]{hyperref}

\floatstyle{ruled}
\newfloat{code}{thp}{lop}
\floatname{code}{Code}

\begin{document}

\newcommand{\vexdifx}{{\tt vex2difx} }
\newcommand{\vd}{{\tt .v2d} }
\newcommand{\vx}{{\tt .vex} }
\newcommand{\defname}{{\em def name} }
\newcommand{\fs}{{\tt /}}
\newcommand{\hy}{{\tt -}}
\newcommand{\us}{{\tt \_}}

\begin{center}
{\Large How does vex2difx work?}

\vspace{10pt}
Walter Brisken

\vspace{10pt}
{\em draft} 2015/08/31
\end{center}


\section{Introduction}

This document was produced to demystify the inner workings of \vexdifx...
The version of \vexdifx covered here is that which will become part of the DiFX 2.5 release series.
This version has had considerable changes since the most recent stable version (2.4.0).

\section{Sources of information}

\vexdifx explicitly takes a single file as its input, the \vd file.
This file must contain a reference to a \vx file (using the {\tt vex} parameter.
Contents of the {\tt .vex} file will be used unless overridden in the {\tt .v2d} file.
Additional optional files referenced by the {\tt .v2d} file will be discussed as needed.
Some bits of information (e.g., names of files to correlate) can only be provided by the \vd file, and some information, such as frequency setups, can only be provided by the \vx file.
Little attention will be paid to these cases where the source of information is unambiguous.
The \vexdifx wiki page and vex documentation should be sufficient to understand these cases.
The more complicated options are those where settings in the \vd file are used to override those in the \vx file; that will be the primary focus of this document.

Note that as of now the only version of the vex format that is supported is vex 1.5.
This version of the standard itself is not fully capable of conveying many details of modern VLBI systems, including the VDIF format, some recording systems, and all eVLBI.
The lack of unambiguous support by vex 1.5 means that \vexdifx had to adopt some conventions and make some assumptions.
\vexdifx aims to warn the user when assumptions are being made.

\section{Output files}

\vexdifx generates three output files for each job created.
See Sec.~\ref{sec:break} for information on how \vexdifx breaks an observation into multiple jobs.
The {\tt .input} file is the master file for each DiFX job.
It is passed to {\tt mpifxcorr} and has reference to other files.
The {\tt .calc} file is primarily used as input to generate the delay model {\tt .im} file that {\tt mpifxcorr} requires.
Finally a {\tt .flag} file is written.
This file is used to mask data that may be inadvertently correlated but that is irrelevant to the particular job.
This is important espeically when the observing antenna array is broken into subarrays, each observing different sources.
It is also important in cases where an observation is interrupted but where data for the interrupting observation is accessible to the correlator.
See the DiFX reference manual \url{...} for more information.

\section{Details}

The sections below detail how various bits of information get conveyed to the DiFX input files.

\subsection{Frequency tunings}

At the moment the only way to specify frequency tunings is in the \vx file within a \$FREQ block.

\subsection{Data format}

The data format is probably the most confusing and complicated parameter.
The format, in the context of DiFX, consists of a number of more elementary parameters, some of which are not applicable to every format.
Things are further complicated by the fact that \vexdifx allows some of these parts to be determined by the \vx file and some to be set by the \vd file.
Various parts of the format include
\begin{itemize}
\item {\em class}: VDIF, Mark5B, VLBA, Mark4, and a few other variants and lesser used options.
\item {\em bits}: number of bits per sample, often equal to 1 or 2.
\item {\em chans}: the number of baseband channels has an impact on the layout of data so is considered part of the data format.  Note that not all of the channels need be correlated, but DiFX needs to know which channels are present in any case.  See more on this in Sec.~\ref{sec:channels}
\item {\em fanout}: used only in older formats (VLBA and Mark4).  From the DiFX perspective this describes the internal ordering of channel data in a data stream.
\item {\em threads}: currently used only for the VDIF format.  Thread is a named identifier for a channel set within a VDIF stream.  One or more channel can be contained in each thread.
\item {\em sampling type}: in most cases today this is ``real''.  Two forms of complex sampling (single side-band and double-sideband) are also supported, but only for VDIF format.
\item {\em size}: the size in bytes of one frame of data (currently required only for VDIF).  Unless otherwise noted, {\em size} includes the framing overhead.
\item {\em rate}: the data rate (excluding framing overhead), in Mbps.
\end{itemize}

The \vx file does not provide all of this information in a single place, but instead it can (often) be constructed by combining information from various places in the \vx file, but mostly from the {\tt \$TRACKS} block.
Typically some parts of the format will be specified by the {\tt track\_frame\_format} parameter in the {\tt \$TRACKS} block.
\vexdifx is able to parse a wide variety of values for this parameter, most of which contain more information than just {\tt format class}.
The wide variety of syntaxes that are supported is in response to a wide variety of specifications and informal usage that has accumulated over the years.
Fortunately, there are no known cases where ambiguity arises.
The complete list of possible values is shown in Table~\ref{tab:format}.

\begin{table}
\begin{center}
\caption{
Allowed format specifiers.
The numbering is consistent with internal details of the format matching code in \vexdifx.
The {\em class} parameter is case insensitive.
Certain shorthands are allowed: {\tt MARK5B} format can be written as {\tt MK5B} and {\tt MARK4} can be written as {\tt MKIV}.
See Table~\ref{tab:vdifclass} for specifying variations of the {\tt VDIF} format class.
}
\label{tab:format}
\begin{tabular}{llll}
\# & Syntax & Example & Notes \\
\hline
0 & {\em class} & {\tt MARK5B} & \\
1 & {\em class}\fs{\em threads}\fs{\em size}\fs{\em bits} & {\tt VDIF/0:1:2:3/5032/2} & For VDIF formats only. \\
2 & {\em class}\fs{\em size}\fs{\em bits} & {\tt VDIF/5032/2} & For VDIF formats only. \\
3 & {\em class} {\em size} & {\tt VDIF5032} & This syntax is discouraged. \\
4 & {\em class}{\tt 1\_}{\em fanout} & {\tt VLBA1\_4} & For VLBA and Mark4 formats only. \\
5 & {\em class}\us{\em size}\hy{\em rate}\hy{\em chans}\hy{\em bits} & {\tt VDIF\_5000-2048-16-2} & VDIF only. \\
  & & & {\em size} here excludes frame headers. \\
6 & {\em class}\hy{\em rate}\hy{\em chans}\hy{\em bits} & {\tt MARK5B-1024-8-2} & \\
7 & {\em class}{\tt 1\_}{\em fanout}\hy{\em rate}\hy{\em chans}\hy{\em bits} & {\tt VLBA1\_4-512-8-2} & For VLBA and Mark4 formats only. \\
8a & {\em class}\fs{\em bits} & {\tt MARK5B/2} & Assumes {\em bits} $\le 32$.  Use is discouraged. \\
8b & {\em class}\fs{\em size} & {\tt VDIF/5032} & Assumes {\em size} $> 32$.  Use is discouraged. \\
8c & {\em class}\hy{\em bits} & {\tt MARK5B-2} & Assumes {\em bits} $\le 32$.  Use is discouraged. \\
8d & {\em class}\hy{\em size} & {\tt VDIF-5032} & Assumes {\em size} $> 32$.  Use is discouraged. \\
\end{tabular}
\end{center}
\end{table}

\begin{table}
\begin{center}
\caption{
Supported classes for various VDIF formats.
}
\label{tab:vdifclass}
\begin{tabular}{p{3.5cm}p{12cm}}
Class & Description \\
\hline
{\tt VDIF} & Can describe any non-legacy VDIF.  Usually defaults to single-thread unless threads are explicitly defined.  See note on canonical threads (Sec.~\ref{sec:canonicalthreads}). \\
{\tt VDIFL} & Specify Legacy VDIF.  DiFX supports legacy VDIF only in single-thread cases. \\
{\tt VDIFC} & Same as {\tt VDIF}, but specifies single-sideband complex sampling. \\
{\tt VDIFD} & Same as {\tt VDIF}, but specifies double-sideband complex sampling. \\
{\tt INTERLACEDVDIF} & Same as {\tt VDIF}, but explicitly forces multiple threads.  The list of threads must be provided through some means. \\
{\tt INTERLACEDVDIFC} & Same as {\tt INTERLACEDVDIF}, but for single-sideband complex sampling. \\
{\tt INTERLACEDVDIFD} & Same as {\tt INTERLACEDVDIF}, but for double-sideband complex sampling. \\
\end{tabular}
\end{center}
\end{table}

Note that behavior is undefined if the \vx file contains conflicting information regarding the format of data.

The subsections below dictate how the various format parts are specified.
Everything in this section is relevant both for single and multiple datastreams per channel.
For details on configuring multiple datastreams, see Sec.~\ref{sec:mds}.

\subsubsection{Format class}

Format class can only be specified in two places: the {\tt track\_frame\_format} parameter in the \vx file's {\tt \$TRACKS} section and the {\tt format} parameter in the \vd file (which can live either in an {\tt ANTENNA} or {\tt DATASTREAM} section).
Collectively these parameters will be called ``format parameters''.
At least one file must provide this information.
If the information is provided in both files, the \vd file will override.
Both the \vx and \vd parameters accept the same set of possible values as enumerated in Table~\ref{tab:format}.
The sources of format class, listed in increasing priority, are:
\begin{enumerate}
\item {\tt track\_frame\_format} statement in a {\tt \$TRACKS} block of the \vx file
\item {\tt format} statement in a {\tt ANTENNA} section of the \vd file
\item {\tt format} statement in a {\tt DATASTREAM} section of the \vd file
\end{enumerate}

\subsubsection{Number of bits}

The number of bits per sample can be specified in the format parameters.
However, if the \vx file has tracks defined in the {\tt \$TRACKS} block, the number of bits is determined by the absense or presense of magnitude tracks: if a magnitude track is found, 2 bits is assumed, otherwise 1 bit is used.
Any format parameter in the \vd file containing number of bits will override that in the \vx file, however it is deduced.
These sources of bits per sample, listed in increasing priority, are:
\begin{enumerate}
\item {\tt track\_frame\_format} statement in a {\tt \$TRACKS} block of the \vx file
\item Existence of magnitude tracks listed in {\tt fanout\_def} statments in a {\tt \$TRACKS} block of the \vx file
\item {\tt format} statement in a {\tt ANTENNA} section of the \vd file
\item {\tt format} statement in a {\tt DATASTREAM} section of the \vd file
\end{enumerate}

\subsubsection{Number of channels}

The number of recorded baseband channels can be specified in the format parameters.
There are two additional sources of channel count within the \vx file: the {\tt format\_def} statements in {\tt \$TRACKS} section and the {\tt chan\_def} statements in the {\tt \$FREQ} block. 
These sources of channel count, listed in increasing priority, are:
\begin{enumerate}
\item {\tt track\_frame\_format} statement in a {\tt \$TRACKS} block of the \vx file
\item Number of {\tt chan\_def} entries in a {\tt \$FREQ} block of the \vx file
\item Number of sign tracks listed in {\tt fanout\_def} statments in a {\tt \$TRACKS} block of the \vx file
\item {\tt format} statement in a {\tt ANTENNA} section of the \vd file
\item {\tt format} statement in a {\tt DATASTREAM} section of the \vd file
\end{enumerate}

\subsubsection{Fanout}

Fanout is relevant only to VLBA and Mark4 formats, and some other very closely related formats\footnote{E.g., Mark3 and the so-called {\tt VLBN} format, which is equivalent to the VLBA format but without modulation.}

Fanout can be taken from three sources

\subsubsection{Canonical VDIF threads} \label{sec:canonicalthreads}

\subsubsection{Format overriding with multiple modes}

In general, format settings made within the \vd file apply to all modes.
It is not possible to explicitly set format parameters separately for each mode, but careful omission of parameters in the \vd file can allow the mode-dependent values from the \vx file to remain in control.
Essentially any format parameter that must change with mode must not be set in the \vd file.

\subsection{Channel ordering} \label{sec:channels}

\subsection{Multiple datastreams} \label{sec:mds}

\subsection{Antenna properties}

As a warning, within the DiFX ecosystem three terms, ``station'', ``antenna'', and ``telescope'' are all used more or less interchangably.

\subsubsection{Antenna name}

By default the antenna name assigned by \vexdifx is the \defname of the corresponding entry in the {\tt \$STATION} block of the \vx file, promoted to all capital letters.
Antenna names can be overridden in the \vd file within a corresponding {\tt ANTENNA} section through the use of the {\tt name} parameter.

\subsubsection{Antenna coordinates}

\subsubsection{Axis offsets}

\subsubsection{Clock models}

\subsubsection{Ephemeris information}

\subsection{Earth orientation parameters}



\section{Job separation} \label{sec:break}

\end{document}
